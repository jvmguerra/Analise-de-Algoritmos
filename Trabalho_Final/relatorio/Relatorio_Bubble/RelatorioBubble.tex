\documentclass[12pt,a4paper,twoside]{report}
% -------------------------------------------------------------------- %
% Pacotes

\usepackage[utf8]{inputenc}
\usepackage[T1]{fontenc}
\usepackage[brazil]{babel}
\usepackage[fixlanguage]{babelbib}
\usepackage[pdftex]{graphicx}      % usamos arquivos pdf/png como figuras
\usepackage{setspace}              % espaçamento flexvel
\usepackage{indentfirst}           % indentação do primeiro parágrafo
\usepackage{makeidx}               % índice remissivo
\usepackage[nottoc]{tocbibind}     % acrescentamos a bibliografia/indice/conteudo no Table of Contents
\usepackage{courier}               % usa o Adobe Courier no lugar de Computer Modern Typewriter
\usepackage{type1cm}               % fontes realmente escaláveis
\usepackage{titletoc}
\usepackage{ucs}
\usepackage[font=small,format=plain,labelfont=bf,up,textfont=it,up]{caption}
\usepackage[usenames,svgnames,dvipsnames]{xcolor}
\usepackage[a4paper,top=2.54cm,bottom=2.0cm,left=2.0cm,right=2.54cm]{geometry} % margens
\usepackage{amsmath} 

\usepackage[pdftex,plainpages=false,pdfpagelabels,pagebackref,colorlinks=true,citecolor=DarkGreen,
linkcolor=NavyBlue,urlcolor=DarkRed,filecolor=green,bookmarksopen=true]{hyperref} % links coloridos
\usepackage[all]{hypcap}                % soluciona o problema com o hyperref e capítulos
\usepackage[square,sort,nonamebreak,comma]{natbib}  % citação bibliográfica alpha
\fontsize{60}{62}\usefont{OT1}{cmr}{m}{n}{\selectfont}
\usepackage{upquote}                    % formata apóstrofes '
\usepackage{textcomp}

% Para formatar corretamente as URLs
\usepackage{url}
% -------------------------------------------------------------------- %
% Cabeçalhos similares ao TAOCP de Donald E. Knuth
\usepackage{fancyhdr}
\pagestyle{fancy}
\fancyhf{}
\renewcommand{\chaptermark}[1]{\markboth{\MakeUppercase{#1}}{}}
\renewcommand{\sectionmark}[1]{\markright{\MakeUppercase{#1}}{}}
\renewcommand{\headrulewidth}{0pt}

% -------------------------------------------------------------------- %
\graphicspath{{./imagens/}}        % caminho das figuras
\frenchspacing                     % arruma o espaço: id est (i.e.) e exempli gratia (e.g.)
\urlstyle{same}                    % URL com o mesmo estilo do texto e no mono-spaced
\makeindex                         % para o índice remissivo
\raggedbottom                      % para no permitir espaços extras no texto
\fontsize{60}{62}\usefont{OT1}{cmr}{m}{n}{\selectfont}
\cleardoublepage
\normalsize

% -------------------------------------------------------------------- %
% Cores para formatação de código
\usepackage{color}
\definecolor{vermelho}{rgb}{0.6,0,0} % para strings
\definecolor{verde}{rgb}{0.25,0.5,0.35} % para comentários
\definecolor{roxo}{rgb}{0.5,0,0.35} % para palavras-chaves
\definecolor{azul}{rgb}{0.25,0.35,0.75} % para strings
\definecolor{cinza-claro}{gray}{0.95}
% -------------------------------------------------------------------- %
% Opções de listagem usados para o código fonte
% Ref: http://en.wikibooks.org/wiki/LaTeX/Packages/Listings



\usepackage{listings}           % para formatar código-fonte (ex. em Java)


\lstset{ %
language=Python,                      % seleciona a linguagem do código
basicstyle=\footnotesize\ttfamily,    % o tamanho da fonte usado no código
commentstyle=\color{verde}\bfseries,  % formatação de comentários
stringstyle=\color{azul},             % formatação de strings
upquote=true,
numbers=left,                   % onde colocar os números de linha
numberstyle=\tiny,  % o tamanho da fonte usada para a numeração das linhas
stepnumber=1,                   % o intervalo entre dois números de linhas. Se for 1, numera cada uma.
numbersep=5pt,                  % how far the line-numbers are from the code
showspaces=false,               % show spaces adding particular underscores
showstringspaces=false,         % underline spaces within strings
showtabs=false,                 % show tabs within strings adding particular underscores
keywordstyle=\color{roxo}\bfseries,
keywordstyle=[1]\color{roxo}\bfseries,
keywordstyle=[2]\color{verde}\bfseries,
frame=b,                   % adds a frame around the code
framerule=0.6pt,
tabsize=2,                      % sets default tabsize to 2 spaces
captionpos=t,                   % sets the caption-position to top
breaklines=true,                % sets automatic line breaking
breakatwhitespace=false,        % sets if automatic breaks should only happen at whitespace
escapeinside={\%*}{*)},         % if you want to add a comment within your code
backgroundcolor=\color[rgb]{1.0,1.0,1.0}, % choose the background color.
rulecolor=\color[rgb]{0.8,0.8,0.8},
extendedchars=true,
xleftmargin=10pt,
xrightmargin=10pt,
framexleftmargin=10pt,
framexrightmargin=10pt,
literate={â}{{\^{a}}}1  % para formatar corretamente os acentos do Português ao usar utf8
    {ê}{{\^{e}}}1
    {ô}{{\^{o}}}1  
    {Â}{{\^{A}}}1
    {Ê}{{\^{E}}}1
    {Ô}{{\^{O}}}1
    {á}{{\'{a}}}1
    {é}{{\'{e}}}1
    {í}{{\'{i}}}1
    {ó}{{\'{o}}}1
    {ú}{{\'{u}}}1
    {Á}{{\'{A}}}1
    {É}{{\'{E}}}1
    {Í}{{\'{I}}}1
    {Ó}{{\'{O}}}1
    {Ú}{{\'{U}}}1
    {à}{{\`{a}}}1
    {À}{{\`{A}}}1
    {ã}{{\~{a}}}1
    {õ}{{\~{o}}}1
    {Ã}{{\~{A}}}1
    {Õ}{{\~{O}}}1
    {ç}{{\c{c}}}1
    {Ç}{{\c{C}}}1
    {ü}{{\"u}}1
    {Ü}{{\"U}}1
}

\renewcommand{\lstlistingname}{Listagem}
\renewcommand{\lstlistlistingname}{Lista de Listagens}

% Definição de novos estilos
\lstdefinestyle{Bash}
    {language=bash,frame=single,numbers=none,basicstyle=\footnotesize\ttfamily,
     morekeywords={cp,mkdir,sudo,tar}}

% Definição de novos ambientes
\lstnewenvironment{terminal}
  {\lstset{style=Bash}}
  {}

\lstnewenvironment{python}
  {\lstset{basicstyle=\scriptsize\ttfamily,
           frame=single,
           frameround=tttt,
           framerule=2pt,
           numbers=none,
           rulecolor=\color{Salmon}}}
  {}

\lstnewenvironment{ide}
  {\lstset{frame=single,
            frameround=tttt,
            numbers=none,
            basicstyle=\ttfamily,
            framerule=2pt,
            rulecolor=\color{CadetBlue}}}
  {}

\lstnewenvironment{latex}
   {\lstset{language=[LaTex]TeX,
            basicstyle=\scriptsize\ttfamily,
            frame=none,
            numbers=none}}
   {}


% Formata o caption da listagem
% \DeclareCaptionFont{blue}{\color{blue}} 

% \captionsetup[lstlisting]{singlelinecheck=false, labelfont={blue}, textfont={blue}}
\usepackage{caption}
\DeclareCaptionFont{white}{\color{white}}
\DeclareCaptionFormat{listing}{\colorbox[cmyk]{0.43, 0.35, 0.35,0.01}{\parbox{\textwidth}{\hspace{15pt}#1#2#3}}}
\captionsetup[lstlisting]{format=listing,labelfont=white,textfont=white, singlelinecheck=false, margin=0pt, font={bf,footnotesize}}

\newcommand{\ListingsPath}{./codigos}
% Inclui o nome do arquivo como Caption 
\newcommand{\filelisting}[2][]{%
    \lstinputlisting[caption={\texttt{\detokenize{#2}}},#1]{\ListingsPath/#2}%
}

% ---------------------------------------------------------------------------- %

% ---------------------------------------------------------------------------- %

\title{Análise de Complexidade de Tempo do Método Bubble Sort}
\date{}
\author{Eduardo Costa de Paiva \\
\texttt{\small \url{eduardocspv@gmail.com}}\\
Frederico Franco Calhau \\
\texttt{\small \url{fredericoffc@gmail.com}}\\
Gabriel Augusto Marson \\
\texttt{\small \url{gabrielmarson@live.com}}\\
\vspace{1cm} \\
Faculdade de Computação \\
Universidade Federal de Uberlândia
}
\date{\today}

\begin{document}
  \maketitle
% -------------------------------------------------------------------- %
% Listas de figuras, tabelas e códigos criadas automaticamente
\listoffigures            
\listoftables            
\lstlistoflistings
% -------------------------------------------------------------------- %

% -------------------------------------------------------------------- %
% Sumário
\tableofcontents    

% Capítulos do trabalho

% cabeçalho para as páginas de todos os capítulos
\fancyhead[RE,LO]{\thesection}

%\singlespacing              % espaçamento simples
\setlength{\parskip}{0.15in} % espaçamento entre paragráfos

\chapter{Introdução}
Este documento foi feito com o intuito de exibir uma análise do algoritmo Bubble Sort 
com relação a tempo. Além disso, será feita uma comparação da curva de tempo do que se espera do
algoritmo, ou seja, $O(n^2)$ com o caso prático. 

\section{Diretório}

Dada a seguinte organização das pastas, utilizamos o arquivo testdriver.py,  executando, uma função conveniente por vez. Para mais informações vá até ao apêndice.

OBS.: É necessário instalar o programa tree pelo terminal. Isso pode ser feito da seguinte maneira.

\begin{terminal}
> sudo apt-get install tree
\end{terminal}

A seguir é mostrada a organização das pastas sendo que os diretórios significativas para o projeto são Codigos e Relatorio além do raíz:
\begin{terminal}
tree --charset=ASCII
.
|-- Codigos
|   |-- Bubble
|   |   |-- BubbleSort.py
|   |   `-- __pycache__
|   |       |-- BubbleSort.cpython-34.pyc
|   |       `-- testeBubble.cpython-34.pyc
|   `-- README.md
|-- memoria.py
|-- monitor.py
|-- Other
|   |-- expfit0.py
|   |-- expfit.py
|   |-- leialprof.py
|   |-- leitura1.py
|   |-- leitura2.py
|   |-- leitura.py
|   `-- logfit.py
|-- Plot
|   |-- plot1.py
|   |-- plot2.py
|   |-- plot3.py
|   `-- plot_tempo.py
|-- __pycache__
|   |-- monitor.cpython-34.pyc
|   `-- tempo.cpython-34.pyc
|-- relatorio
|   |-- imagens
|   |   |-- Bubble
|   |   |   |-- bubble_plot_1.png
|   |   |   `-- bubble_plot_3.png
|   |   `-- Merge
|   |-- Relatorio_Bubble
|   |   |-- RelatorioBubble.aux
|   |   |-- RelatorioBubble.idx
|   |   |-- RelatorioBubble.lof
|   |   |-- RelatorioBubble.log
|   |   |-- RelatorioBubble.lol
|   |   |-- RelatorioBubble.lot
|   |   |-- RelatorioBubble.out
|   |   |-- RelatorioBubble.tex
|   |   `-- RelatorioBubble.toc
|   |-- Relatorio_Merge
|   `-- Resultados
|       |-- Bubble
|       |   `-- tBolha.dat
|       `-- Merge
|-- tempo.py
|-- testdriver.py
|-- testeGeneric.py
`-- testeGeneric.py.lprof

15 directories, 35 files

\end{terminal}

\section{Códigos de programas}
Seguem os códigos utilizados na análise de tempo do algoritmo Bubble Sort.
\begin{enumerate}

\item BubbleSort.py:
Disponível na Listagem~\ref{arq:BubbleSort}.
\lstinputlisting[caption={BubbleSort.py \label{arq:BubbleSort}}]{../../Codigos/Bubble/BubbleSort.py}

\item testeGeneric.py
Disponível na Listagem ~\ref{arq:testeGeneric}
\lstinputlisting[caption={testeGeneric.py \label{arq:testeGeneric}}]{../../testeGeneric.py}

\item monitor.py
Disponível na Listagem~\ref{arq:monitor}
\lstinputlisting[caption={monitor.py \label{arq:monitor}}]{../../monitor.py}


\item testdriver.py
 Referenciado no apêndice~\ref{ap:testdriver}.
\end{enumerate}


\chapter{Imagens}

Se desejar incluir imagens você poderá usar o comando a seguir:

\begin{latex}
\begin{figure}[!ht]
\centering
\includegraphics[scale=0.5]{imagens/bolha1.png}
\caption{Complexidade de tempo do método da bolha \label{fig:1}}
\end{figure}
\end{latex}

O comando anterior gerará a imagem a seguir, dado que o arquivo \texttt{fig1.png} esteja no subdiretório \texttt{imagens}.
\begin{figure}[!h]
\centering
\includegraphics[scale=0.5]{../imagens/Bubble/bubble_plot_1.png}
\caption{Complexidade de tempo do método da bolha \label{fig:1}}
\end{figure}

Note que a Figura~\ref{fig:1} pode ser referenciada em qualquer parte do documento.  Você também pode incluir diretamente outros formatos de imagens tais como o \texttt{jpg}.

\chapter{Tabelas}

Aqui vão alguns exemplos de criação de tabelas.

Vamos criar uma tabela simples com o código a seguir. O resultado é a 
Tabela~\ref{tab:opcode1}

\begin{latex}
\begin{table}[h]
  \centering
  \caption{Alguns opcodes do Dalvik \label{tab:opcode1}}
  \begin{tabular}{lll} \hline
  {\bf Opcode (hex)} & {\bf Nome do opcode} & {\bf Explicação} \\ \hline
  00 & nop & Somente gasta alguns ciclos do processador \\ \hline
  01 & move vx, vy & Move o conteúdo de vy em vx. Ambos os registradores
   devem estar no intervalo de 0 a 255 \\\hline
  \end{tabular}
\end{table}
\end{latex}

\begin{table}[h]
  \centering
  \caption{Alguns opcodes do Dalvik \label{tab:opcode1}}
  \begin{tabular}{lll} \hline
  {\bf Opcode (hex)} & {\bf Nome do opcode} & {\bf Explicação} \\ \hline
  00 & nop & Somente gasta alguns ciclos do processador \\ \hline
  01 & move vx, vy & Move o conteúdo de vy em vx. Ambos os registradores
   devem estar no intervalo de 0 a 255 \\\hline
  \end{tabular}
\end{table}

Note que o texto sob a coluna \texttt{Explicação} é muito longo e a
tabela~\ref{tab:opcode1} ficou mal formatada. Podemos resolver esse problema
usando colunas de largura fixa, conforme mostrado no código a seguir e cujo
resultado é a tabela~\ref{tab:opcode2}.

\begin{latex}
\begin{table}[h]
  \centering
  \caption{Alguns opcodes do Dalvik \label{tab:opcode2}}
  \begin{tabular}{p{1.5cm}lp{9cm}} \hline
  {\bf Opcode (hex)} & {\bf Nome do opcode} & {\bf Explicação} \\ \hline
  00 & nop & Somente gasta alguns ciclos do processador \\ \hline
  01 & move vx, vy & Move o conteúdo de vy em vx. Ambos os registradores
   devem estar no intervalo de 0 a 255 \\\hline
  \end{tabular}
\end{table}
\end{latex}

\begin{table}[h]
  \centering
  \caption{Alguns opcodes do Dalvik \label{tab:opcode2}}
  \begin{tabular}{p{1.5cm}lp{9cm}} \hline
  {\bf Opcode (hex)} & {\bf Nome do opcode} & {\bf Explicação} \\ \hline
  00 & nop & Somente gasta alguns ciclos do processador \\ \hline
  01 & move vx, vy & Move o conteúdo de vy em vx. Ambos os registradores
   devem estar no intervalo de 0 a 255 \\\hline
  \end{tabular}
\end{table}

\chapter{Citações e referências bibliográficas}
Todas as fontes usadas para a confecção de seu relatório devem ser citadas. Isso
inclui livros e documentos da internet, tais como tutoriais e páginas.

\clearpage
\addcontentsline{toc}{part}{Apêndice}
\appendix

\chapter{Códigos extensos \label{ap:testdriver}}
\section{testdriver.py}
\lstinputlisting[label= {arq:testdriver.py}, caption={testdriver.py}] {../../testdriver.py}


\end{document} 